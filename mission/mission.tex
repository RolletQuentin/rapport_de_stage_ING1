\subsection{Mission}

Au cours de mon stage, j'ai entrepris diverses missions au sein de l'association dans le but de mener à bien le projet qui m'a été confié.

Initialement, je vais détailler en profondeur le projet qui m'a été attribué, puis je vais fournir des explications approfondies sur les différentes tâches entreprises pour la réalisation réussie de ce projet.

\subsubsection{Le projet}

Le projet consiste en la création d'une application interne à l'école CY~Tech, visant à améliorer la communication des associations étudiantes concernant les événements qu'elles organisent. Cette application offre aux étudiants la possibilité de s'inscrire aisément aux événements et de rester informés plus facilement de ce qui se passe autour d'eux.

Le projet est subdivisé en deux sous-projets : le développement d'une application web et la création d'une API.

L'application web fait office d'interface graphique entre l'utilisateur et l'API, assurant la liaison entre la base de données et l'interface utilisateur. Nous avons fait le choix de diviser le projet en deux parties distinctes pour garantir une maintenance optimale des applications et permettre à plusieurs applications web de bénéficier de ce service.

Plusieurs fonctionnalités essentielles sont prévues pour les applications :
\begin{itemize}
	\item Création et gestion de comptes utilisateurs ;
	\item Abonnement aux associations ;
	\item Participation aux événements ;
	\item Création d'événements ponctuels ou récurrents.
\end{itemize}

Il est impératif que les applications soient hautement sécurisées, tout en garantissant une maintenance aisée et une extensibilité optimale.
\subsubsection{Détails}