\section{Bilan}

%« Et si c’était à refaire je changerai quoi et pourquoi ? »

% Qu’est-ce que j’ai appris sur les quatre axes suivants : Profession, entreprise, savoir technique et savoir être et moi-même.
% Il également important de réfléchir sur les conditions de travail et du futur métier.

Ce stage s'est avéré être une réelle opportunité pour moi de découvrir le métier de développeur fullstack ainsi que les aspects liés au travail en tant que freelance. En effet, la liberté que m'a accordée mon maître de stage en ce qui concerne la mission, ainsi que la nature de mon travail au sein de l'association, ont créé un contexte similaire à une relation client / développeur. Dans cette perspective, l'association jouait le rôle du client, même si les membres de l'association m'ont apporté leur aide sur certains points.

Cette expérience a renforcé ma vision du métier de développeur et a confirmé mon désir de devenir développeur freelance dans un avenir proche. Cependant, j'aspire également à effectuer un stage plus structuré au sein d'une entreprise, afin de me confronter à la collaboration sur des projets et de mieux appréhender l'environnement professionnel du point de vue d'un développeur. Cette expérience me permettrait d'acquérir de nouvelles approches de programmation, de découvrir des méthodes de collaboration innovantes et d'obtenir une perspective plus globale du métier de développeur, tout en ayant l'opportunité d'interagir avec des développeurs seniors qui pourraient partager leur expérience.

\medskip

Ce stage m'a permis de consolider ma vision du rôle d'un ingénieur en informatique. En effet, la majeure partie du travail ne consiste pas uniquement à coder en continu, mais plutôt à apprendre, à lire abondamment la documentation, à réfléchir sur la manière de programmer tout en anticipant les éventuelles conséquences, à collaborer avec d'autres programmeurs et à cogiter ensemble sur des solutions pour résoudre des problèmes. Ce qui me fascine le plus dans ce métier, c'est la variété des tâches et la possibilité de trouver plusieurs solutions à un même problème.

J'ai également acquis une solide expertise technique en ce qui concerne le développement d'applications web et d'API, grâce à l'apprentissage de nombreuses nouvelles technologies.

En fin de compte, ce stage m'a permis de confirmer ma propre vision de moi-même. J'ai maintenu ma motivation et mon enthousiasme tout au long de cette expérience, et j'ai investi beaucoup de temps et d'efforts dans mon travail. Quand je m'engage dans quelque chose que j'aime, je peux y consacrer de nombreuses heures avec plaisir.

\medskip

En ce qui concerne les conditions de travail, le métier d'ingénieur en informatique ne suscite aucune plainte de ma part. Comparé à mon mi-temps dans une aire d'autoroute, ainsi qu'à mes expériences d'intérim pendant mes études, ce métier est moins exigeant physiquement, et le sentiment d'accomplissement est bien plus palpable lorsqu'un projet est finalisé. Cependant, il est important de noter que ce travail requiert un investissement en termes d'heures considérable, que ce soit en travaillant intensément sur un projet en une semaine ou en consacrant du temps à l'apprentissage continu des nouvelles versions des technologies utilisées, ainsi qu'à la veille technologique pour rester informé des avancées du domaine.