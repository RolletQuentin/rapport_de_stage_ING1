% L’introduction est le lieu dans lequel le lecteur va découvrir le contenu du rapport et le lieu dans lequel l’étudiant doit expliquer les raisons qui ont motivé son choix du stage. Les choix techniques et humains et autres que l’obligation de faire un stage. L’introduction doit décrire très succinctement le déroulement général et expliquer le plan adopté pour le rapport.

\section{Introduction}

Dans le cadre du stage de première année du cycle d'ingénieur, j'ai effectué un stage dans une association d'informatique, la Corpauration. Cette association propose aux étudiants de CY~Tech différents événements sur le thème de l'informatique, des jeux-vidéos et des jeux de sociétés, mais leur proposent aussi diverses applications permettant de faciliter la vie étudiante au sein de l'école.

Dans ce contexte, la Corpauration veut proposer une nouvelle application aux étudiants dès la rentrée 2023 : une application leur permettant d'accéder à tout les événements proposés par les associations de l'école, et de s'y inscrire facilement.

Étant membre actif de la Corpauration, et plus généralement membre actif de la vie associative de l'école CY~Tech, j'ai souhaité m'engager à plein temps dans la réalisation de cette application, qui est pour moi très importante puisqu'elle permettra d'améliorer la communication des associations de l'école. C'est pour ces raisons que j'ai demandé à la Corpauration de me prendre comme stagiaire.

\medskip

Je suis alors chargé de créer cette application en développant une API en Kotlin en utilisant le framework Quarkus, et en développant une application web en utilisant le célèbre framework Vuetify de Vue, lui même un framework Javascript.

Pour cela, il faut faire attention à écrire une documentation solide pour que l'application soit maintenue dans le temps ainsi qu'effectuer des tests unitaires et des tests d'intégration pour s'assurer du bon fonctionnement de l'application.

\medskip

DÉCRIRE LES RÉSULTATS OBTENUS -> APP FINALE FONCTIONNALITÉS ?