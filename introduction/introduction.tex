% L’introduction est le lieu dans lequel le lecteur va découvrir le contenu du rapport et le lieu dans lequel l’étudiant doit expliquer les raisons qui ont motivé son choix du stage. Les choix techniques et humains et autres que l’obligation de faire un stage. L’introduction doit décrire très succinctement le déroulement général et expliquer le plan adopté pour le rapport.

\section{Introduction}

Dans le cadre du stage de première année du cycle d'ingénieur, j'ai effectué un stage au sein de l'association d'informatique "Corpauration". Cette association propose aux étudiants de CY~Tech différents événements axés sur l'informatique, les jeux vidéo et les jeux de société, tout en leur offrant diverses applications visant à faciliter leur vie étudiante au sein de l'école.

Dans ce contexte, la Corpauration ambitionne de lancer une nouvelle application dès la rentrée 2023. Cette application permettra aux étudiants d'accéder à l'ensemble des événements proposés par les associations de l'école et de s'y inscrire en toute simplicité.

En tant que membre actif de la Corpauration et de la vie associative de l'école CY~Tech, j'ai fait le choix de m'investir pleinement dans la conception de cette application. Celle-ci revêt une importance particulière à mes yeux, car elle contribuera à améliorer la communication entre les différentes associations de l'école. C'est dans cette optique que j'ai sollicité la Corpauration pour réaliser mon stage au sein de leur structure.

\medskip

Ma mission consiste donc à développer une application répondant à ces besoins. Pour ce faire, je suis chargé de créer une API en Kotlin en utilisant le framework Quarkus, ainsi que de concevoir une application web à l'aide du célèbre framework Vuetify de Vue, lui-même basé sur le langage JavaScript.

Dans cette démarche, il est essentiel de rédiger une documentation complète, garantissant la pérennité de l'application. Parallèlement, je devrai effectuer des tests unitaires et des tests d'intégration afin de vérifier le bon fonctionnement de chaque composant de l'application.

\medskip

DÉCRIRE LES RÉSULTATS OBTENUS -> APP FINALE FONCTIONNALITÉS ?