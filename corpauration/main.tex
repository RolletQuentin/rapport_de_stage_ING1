\section{L'organisme d'accueil}

% Dans cette partie, l’étudiant décrit le contexte du stage. Il présente son institution d’accueil. Cela permet de mettre en évidence la fiche d'identité de l’institution d’accueil, son statut (public /privé), son historique, ses objectifs, son fonctionnement, son organisation, son public, ou encore ses principaux concurrents. Ensuite, est présenté, le service ou département auquel l'étudiant était affecté. Il est primordial que le stagiaire soit en mesure d’en connaître les objectifs, le fonctionnement, ainsi que sa position dans l’équipe.

% Un paragraphe devra décrire comment l’entreprise gère le développement durable.

\subsection{Présentation}

La Corpauration est une association de l'école d'ingénieur CY~Tech. Ce sont des étudiants de cette école qui constitue le bureau, et la plupart des membres de l'association font aussi parti de cette école.

\medskip

La Corpauration possède plusieurs pôles :
\begin{itemize}
	\item Pauwer~Up~!;
	\item AIR-EISTI;
	\item VEISTI'JEU;
	\item Pau~de~Kholles. 
\end{itemize}

\subsection{Contexte du stage}

\input{"./corpauration/pauwerup"}

\subsection{Développement durable}

Comme la Corpauration est une association étudiante centré sur les jeux et le développement de projets, elle n'est que très peu ancrée dans le développement durable.

Cependant, elle fait quand même quelques efforts aux sujet de l'environnement. Elle fait par exemple le tri dans les poubelles de ses locaux, et elle propose chaque année une Coding~Night sur l'impact de la création d'applications sur l'environnement.

Dans ce contexte, elle essaie au mieux d'utiliser des technologies qui utilisent le moins de ressources possibles afin de diminuer son impact environnemental.