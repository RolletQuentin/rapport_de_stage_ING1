% La conclusion permet de faire un bilan des faits majeurs qui ont déjà été mentionnés dans les chapitres. Cependant, elle est également l'occasion d’expliquer si le stage a pu valider le projet professionnel de l’étudiant ou non. Elle précise les chemins possibles et détermine si le stage était une réussite ou sans plus. Elle débouche sur des perspectives d'avenir.

\section{Conclusion}

En conclusion, mon stage au sein de l'association Corpauration m'a offert une expérience enrichissante et formatrice dans le domaine du développement fullstack. Ce parcours m'a permis d'acquérir des compétences techniques approfondies, de découvrir les rouages du métier de développeur freelance et de renforcer ma perception du rôle d'ingénieur en informatique.

Au cours de cette expérience, j'ai eu l'opportunité de participer à la création et au développement d'une application web et d'une API, conçues pour faciliter la gestion et la promotion des événements étudiants. En adoptant une approche axée sur les micro-fonctionnalités, j'ai pu développer des fonctionnalités spécifiques de manière modulaire, ce qui a permis d'offrir rapidement une version fonctionnelle de l'application tout en laissant place à des améliorations continues.

La communication a joué un rôle essentiel tout au long de ce stage. Travaillant au sein d'une association, j'ai pu constater l'importance d'une communication efficace avec les membres de l'association, ainsi qu'avec mon maître de stage et le vice-président. La distance géographique a été atténuée grâce à l'utilisation d'outils tels que Discord et GitHub, ce qui a favorisé une collaboration productive malgré les contraintes du télétravail.

L'apprentissage de nouvelles technologies a été un défi stimulant. J'ai dû maîtriser diverses technologies telles que Vue.js, Vuetify, Quarkus et Hibernate, en recherchant des solutions innovantes pour résoudre les problèmes techniques rencontrés. Bien que certains manques de documentation aient pu ralentir mon apprentissage, la détermination et la recherche de solutions alternatives m'ont permis de progresser et de relever ces défis.

L'expérience avec le développement réactif de l'API a constitué un point d'apprentissage majeur. La mise en œuvre d'une application capable de traiter plusieurs requêtes en parallèle, tout en évitant les erreurs d'exécution asynchrone, a renforcé ma compréhension des principes fondamentaux de la programmation réactive.

Ce stage m'a conforté dans ma passion pour le développement et m'a offert un aperçu concret du quotidien d'un ingénieur en informatique. Les longues heures d'apprentissage, la recherche de solutions, la collaboration avec d'autres programmeurs et la conception d'une application fonctionnelle ont révélé le caractère multifacette et stimulant de ce métier.

En conclusion, ce stage a été une expérience précieuse qui a renforcé mes compétences techniques, élargi ma compréhension du métier de développeur et confirmé ma détermination à poursuivre une carrière dans le domaine. Je suis reconnaissant envers l'association Corpauration pour cette opportunité et je suis impatient d'appliquer les enseignements tirés de cette expérience dans mes futurs projets.